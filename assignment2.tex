\documentclass[]{article}
\usepackage{lmodern}
\usepackage{amssymb,amsmath}
\usepackage{ifxetex,ifluatex}
\usepackage{fixltx2e} % provides \textsubscript
\ifnum 0\ifxetex 1\fi\ifluatex 1\fi=0 % if pdftex
  \usepackage[T1]{fontenc}
  \usepackage[utf8]{inputenc}
\else % if luatex or xelatex
  \ifxetex
    \usepackage{mathspec}
  \else
    \usepackage{fontspec}
  \fi
  \defaultfontfeatures{Ligatures=TeX,Scale=MatchLowercase}
\fi
% use upquote if available, for straight quotes in verbatim environments
\IfFileExists{upquote.sty}{\usepackage{upquote}}{}
% use microtype if available
\IfFileExists{microtype.sty}{%
\usepackage{microtype}
\UseMicrotypeSet[protrusion]{basicmath} % disable protrusion for tt fonts
}{}
\usepackage[margin=1in]{geometry}
\usepackage{hyperref}
\hypersetup{unicode=true,
            pdftitle={Assignment2 Solution},
            pdfauthor={TA: Zhijin Zhou},
            pdfborder={0 0 0},
            breaklinks=true}
\urlstyle{same}  % don't use monospace font for urls
\usepackage{color}
\usepackage{fancyvrb}
\newcommand{\VerbBar}{|}
\newcommand{\VERB}{\Verb[commandchars=\\\{\}]}
\DefineVerbatimEnvironment{Highlighting}{Verbatim}{commandchars=\\\{\}}
% Add ',fontsize=\small' for more characters per line
\usepackage{framed}
\definecolor{shadecolor}{RGB}{248,248,248}
\newenvironment{Shaded}{\begin{snugshade}}{\end{snugshade}}
\newcommand{\KeywordTok}[1]{\textcolor[rgb]{0.13,0.29,0.53}{\textbf{#1}}}
\newcommand{\DataTypeTok}[1]{\textcolor[rgb]{0.13,0.29,0.53}{#1}}
\newcommand{\DecValTok}[1]{\textcolor[rgb]{0.00,0.00,0.81}{#1}}
\newcommand{\BaseNTok}[1]{\textcolor[rgb]{0.00,0.00,0.81}{#1}}
\newcommand{\FloatTok}[1]{\textcolor[rgb]{0.00,0.00,0.81}{#1}}
\newcommand{\ConstantTok}[1]{\textcolor[rgb]{0.00,0.00,0.00}{#1}}
\newcommand{\CharTok}[1]{\textcolor[rgb]{0.31,0.60,0.02}{#1}}
\newcommand{\SpecialCharTok}[1]{\textcolor[rgb]{0.00,0.00,0.00}{#1}}
\newcommand{\StringTok}[1]{\textcolor[rgb]{0.31,0.60,0.02}{#1}}
\newcommand{\VerbatimStringTok}[1]{\textcolor[rgb]{0.31,0.60,0.02}{#1}}
\newcommand{\SpecialStringTok}[1]{\textcolor[rgb]{0.31,0.60,0.02}{#1}}
\newcommand{\ImportTok}[1]{#1}
\newcommand{\CommentTok}[1]{\textcolor[rgb]{0.56,0.35,0.01}{\textit{#1}}}
\newcommand{\DocumentationTok}[1]{\textcolor[rgb]{0.56,0.35,0.01}{\textbf{\textit{#1}}}}
\newcommand{\AnnotationTok}[1]{\textcolor[rgb]{0.56,0.35,0.01}{\textbf{\textit{#1}}}}
\newcommand{\CommentVarTok}[1]{\textcolor[rgb]{0.56,0.35,0.01}{\textbf{\textit{#1}}}}
\newcommand{\OtherTok}[1]{\textcolor[rgb]{0.56,0.35,0.01}{#1}}
\newcommand{\FunctionTok}[1]{\textcolor[rgb]{0.00,0.00,0.00}{#1}}
\newcommand{\VariableTok}[1]{\textcolor[rgb]{0.00,0.00,0.00}{#1}}
\newcommand{\ControlFlowTok}[1]{\textcolor[rgb]{0.13,0.29,0.53}{\textbf{#1}}}
\newcommand{\OperatorTok}[1]{\textcolor[rgb]{0.81,0.36,0.00}{\textbf{#1}}}
\newcommand{\BuiltInTok}[1]{#1}
\newcommand{\ExtensionTok}[1]{#1}
\newcommand{\PreprocessorTok}[1]{\textcolor[rgb]{0.56,0.35,0.01}{\textit{#1}}}
\newcommand{\AttributeTok}[1]{\textcolor[rgb]{0.77,0.63,0.00}{#1}}
\newcommand{\RegionMarkerTok}[1]{#1}
\newcommand{\InformationTok}[1]{\textcolor[rgb]{0.56,0.35,0.01}{\textbf{\textit{#1}}}}
\newcommand{\WarningTok}[1]{\textcolor[rgb]{0.56,0.35,0.01}{\textbf{\textit{#1}}}}
\newcommand{\AlertTok}[1]{\textcolor[rgb]{0.94,0.16,0.16}{#1}}
\newcommand{\ErrorTok}[1]{\textcolor[rgb]{0.64,0.00,0.00}{\textbf{#1}}}
\newcommand{\NormalTok}[1]{#1}
\usepackage{graphicx,grffile}
\makeatletter
\def\maxwidth{\ifdim\Gin@nat@width>\linewidth\linewidth\else\Gin@nat@width\fi}
\def\maxheight{\ifdim\Gin@nat@height>\textheight\textheight\else\Gin@nat@height\fi}
\makeatother
% Scale images if necessary, so that they will not overflow the page
% margins by default, and it is still possible to overwrite the defaults
% using explicit options in \includegraphics[width, height, ...]{}
\setkeys{Gin}{width=\maxwidth,height=\maxheight,keepaspectratio}
\IfFileExists{parskip.sty}{%
\usepackage{parskip}
}{% else
\setlength{\parindent}{0pt}
\setlength{\parskip}{6pt plus 2pt minus 1pt}
}
\setlength{\emergencystretch}{3em}  % prevent overfull lines
\providecommand{\tightlist}{%
  \setlength{\itemsep}{0pt}\setlength{\parskip}{0pt}}
\setcounter{secnumdepth}{0}
% Redefines (sub)paragraphs to behave more like sections
\ifx\paragraph\undefined\else
\let\oldparagraph\paragraph
\renewcommand{\paragraph}[1]{\oldparagraph{#1}\mbox{}}
\fi
\ifx\subparagraph\undefined\else
\let\oldsubparagraph\subparagraph
\renewcommand{\subparagraph}[1]{\oldsubparagraph{#1}\mbox{}}
\fi

%%% Use protect on footnotes to avoid problems with footnotes in titles
\let\rmarkdownfootnote\footnote%
\def\footnote{\protect\rmarkdownfootnote}

%%% Change title format to be more compact
\usepackage{titling}

% Create subtitle command for use in maketitle
\newcommand{\subtitle}[1]{
  \posttitle{
    \begin{center}\large#1\end{center}
    }
}

\setlength{\droptitle}{-2em}
  \title{Assignment2 Solution}
  \pretitle{\vspace{\droptitle}\centering\huge}
  \posttitle{\par}
  \author{TA: Zhijin Zhou}
  \preauthor{\centering\large\emph}
  \postauthor{\par}
  \predate{\centering\large\emph}
  \postdate{\par}
  \date{4/28/2018}


\begin{document}
\maketitle

\subsection{Assignment2}\label{assignment2}

The file Housing.csv contains information on over 500 census tracts in
Boston, where for each tract multiplevariables are recorded. The last
column (CAT.MEDV) was derived from MEDV (the median value), such that it
obtains the value ``High'' if MEDV \textgreater{} 30 and ``Low''
otherwise. Consider the goal of predicting CAT.MEDV of a tract, given
the information in the first 12 columns. Use R to analyze the data.

\begin{Shaded}
\begin{Highlighting}[]
\NormalTok{housing.df <-}\StringTok{ }\KeywordTok{read.csv}\NormalTok{(}\StringTok{"Housing.csv"}\NormalTok{)}
\end{Highlighting}
\end{Shaded}

\subsubsection{Question 1}\label{question-1}

Partition the data into 60\% Training and 40\% Validation with a seed of
5.

\begin{Shaded}
\begin{Highlighting}[]
\KeywordTok{set.seed}\NormalTok{(}\DecValTok{5}\NormalTok{)}
\NormalTok{train.index <-}\StringTok{ }\KeywordTok{sample}\NormalTok{(}\DecValTok{1}\OperatorTok{:}\KeywordTok{nrow}\NormalTok{(housing.df), }\FloatTok{0.6}\OperatorTok{*}\KeywordTok{nrow}\NormalTok{(housing.df))}
\NormalTok{train.df <-}\StringTok{ }\NormalTok{housing.df[train.index, ]}
\NormalTok{valid.df <-}\StringTok{ }\NormalTok{housing.df[}\OperatorTok{-}\NormalTok{train.index,]}
\end{Highlighting}
\end{Shaded}

\subsubsection{Question 2}\label{question-2}

Use the training set to perform a k-NN classification with the first 12
predictors (ignore the MEDV column), trying values of k from 1 to 5.
Make sure to first normalize the data. What is the best k chosen? What
does this k mean?

\paragraph{Step 1: normalize the data}\label{step-1-normalize-the-data}

\begin{Shaded}
\begin{Highlighting}[]
\CommentTok{# initialize normalized training, validation data, complete data frames to originals}

\NormalTok{train.norm.df <-}\StringTok{ }\NormalTok{train.df}
\NormalTok{valid.norm.df <-}\StringTok{ }\NormalTok{valid.df}
\NormalTok{housing.norm.df <-}\StringTok{ }\NormalTok{housing.df}

\CommentTok{#load package}
\KeywordTok{library}\NormalTok{(caret)}
\NormalTok{norm.values <-}\StringTok{ }\KeywordTok{preProcess}\NormalTok{(train.df[, }\DecValTok{1}\OperatorTok{:}\DecValTok{12}\NormalTok{], }\DataTypeTok{method=}\KeywordTok{c}\NormalTok{(}\StringTok{"center"}\NormalTok{, }\StringTok{"scale"}\NormalTok{))}

\CommentTok{#normalize data}
\NormalTok{train.norm.df[, }\DecValTok{1}\OperatorTok{:}\DecValTok{12}\NormalTok{] <-}\StringTok{ }\KeywordTok{predict}\NormalTok{(norm.values, train.df[, }\DecValTok{1}\OperatorTok{:}\DecValTok{12}\NormalTok{])}
\NormalTok{valid.norm.df[, }\DecValTok{1}\OperatorTok{:}\DecValTok{12}\NormalTok{] <-}\StringTok{ }\KeywordTok{predict}\NormalTok{(norm.values, valid.df[, }\DecValTok{1}\OperatorTok{:}\DecValTok{12}\NormalTok{])}
\end{Highlighting}
\end{Shaded}

\paragraph{\texorpdfstring{Step 2: Perform knn analysis for different
values of k using \texttt{knn}
function.}{Step 2: Perform knn analysis for different values of k using knn function.}}\label{step-2-perform-knn-analysis-for-different-values-of-k-using-knn-function.}

\begin{Shaded}
\begin{Highlighting}[]
\CommentTok{# initialize a data frame with two columns: k, and accuracy to store the accuracy for each k.}
\NormalTok{accuracy.df <-}\StringTok{ }\KeywordTok{data.frame}\NormalTok{(}\DataTypeTok{k =} \DecValTok{1}\OperatorTok{:}\DecValTok{5}\NormalTok{, }\DataTypeTok{accuracy =} \KeywordTok{rep}\NormalTok{(}\DecValTok{0}\NormalTok{, }\DecValTok{5}\NormalTok{))}

\CommentTok{# compute knn for different k on validation.}
\KeywordTok{library}\NormalTok{(FNN)}
\ControlFlowTok{for}\NormalTok{(i }\ControlFlowTok{in} \DecValTok{1}\OperatorTok{:}\DecValTok{5}\NormalTok{) \{}
  \CommentTok{# use ?knn to find out more information}
  \CommentTok{# It worth noting that the input argument cl must be a factor!}
\NormalTok{  knn.pred <-}\StringTok{ }\KeywordTok{knn}\NormalTok{(train.norm.df[, }\DecValTok{1}\OperatorTok{:}\DecValTok{12}\NormalTok{], valid.norm.df[, }\DecValTok{1}\OperatorTok{:}\DecValTok{12}\NormalTok{], }
                  \DataTypeTok{cl =}\NormalTok{ train.norm.df[, }\DecValTok{14}\NormalTok{], }\DataTypeTok{k =}\NormalTok{ i)}
\NormalTok{  accuracy.df[i, }\DecValTok{2}\NormalTok{] <-}\StringTok{ }\KeywordTok{confusionMatrix}\NormalTok{(knn.pred, valid.norm.df[, }\DecValTok{14}\NormalTok{])}\OperatorTok{$}\NormalTok{overall[}\DecValTok{1}\NormalTok{] }
\NormalTok{\}}

\CommentTok{#print out the accuracy matrix}
\NormalTok{accuracy.df}
\end{Highlighting}
\end{Shaded}

\begin{verbatim}
##   k  accuracy
## 1 1 0.9261084
## 2 2 0.9359606
## 3 3 0.8916256
## 4 4 0.9064039
## 5 5 0.8768473
\end{verbatim}

The best k is 2. This means that using the 2 nearest records gives the
best prediction result.

\subsubsection{Question 3}\label{question-3}

Predict the CAT.MEDV for a tract with new information, using the best k.
\#\#\#\# Step 1: First load the

\begin{Shaded}
\begin{Highlighting}[]
\CommentTok{# Predict for the new household using the best k (=2)}
\NormalTok{new.df <-}\StringTok{ }\KeywordTok{data.frame}\NormalTok{(}\DataTypeTok{CRIM =} \FloatTok{0.2}\NormalTok{, }\DataTypeTok{ZN =} \DecValTok{0}\NormalTok{, }\DataTypeTok{INDUS =} \DecValTok{7}\NormalTok{, }\DataTypeTok{CHAS =} \DecValTok{0}\NormalTok{, }\DataTypeTok{NOX =} \FloatTok{0.538}\NormalTok{, }\DataTypeTok{RM =} \DecValTok{6}\NormalTok{, }\DataTypeTok{AGE =} \DecValTok{62}\NormalTok{, }\DataTypeTok{DIS =} \FloatTok{4.7}\NormalTok{, }\DataTypeTok{RAD =} \DecValTok{4}\NormalTok{, }\DataTypeTok{TAX =} \DecValTok{307}\NormalTok{, }\DataTypeTok{PTRATIO =} \DecValTok{21}\NormalTok{, }\DataTypeTok{LSTAT =} \DecValTok{10}\NormalTok{)}
\end{Highlighting}
\end{Shaded}

\paragraph{Step 2: NORMALIZE the new
record}\label{step-2-normalize-the-new-record}

\begin{Shaded}
\begin{Highlighting}[]
\NormalTok{new.norm.df <-}\StringTok{ }\KeywordTok{predict}\NormalTok{(norm.values, new.df)}
\end{Highlighting}
\end{Shaded}

\paragraph{Step 3: Predict the outcome using the best
k.}\label{step-3-predict-the-outcome-using-the-best-k.}

\begin{Shaded}
\begin{Highlighting}[]
\NormalTok{knn.pred.new <-}\StringTok{ }\KeywordTok{knn}\NormalTok{(train.norm.df[, }\DecValTok{1}\OperatorTok{:}\DecValTok{12}\NormalTok{], new.norm.df, }
                    \DataTypeTok{cl =}\NormalTok{ train.norm.df[, }\DecValTok{14}\NormalTok{], }\DataTypeTok{k =} \DecValTok{2}\NormalTok{)}
\CommentTok{# See the result}
\NormalTok{knn.pred.new[}\DecValTok{1}\NormalTok{]}
\end{Highlighting}
\end{Shaded}

\begin{verbatim}
## [1] Low
## Levels: Low
\end{verbatim}

\subsubsection{Question 4}\label{question-4}

If we use the above k-NN algorithm (with the best k) to predict CAT.MEDV
for the training data, what would be the confusion matrix of the
training dataset? What is the training accuracy?

\begin{Shaded}
\begin{Highlighting}[]
\NormalTok{knn.pred.train <-}\StringTok{ }\KeywordTok{knn}\NormalTok{(train.norm.df[, }\DecValTok{1}\OperatorTok{:}\DecValTok{12}\NormalTok{], train.norm.df[, }\DecValTok{1}\OperatorTok{:}\DecValTok{12}\NormalTok{], }
                      \DataTypeTok{cl =}\NormalTok{ train.norm.df[, }\DecValTok{14}\NormalTok{], }\DataTypeTok{k =} \DecValTok{2}\NormalTok{)}
\CommentTok{#Confusion Matrix}
\KeywordTok{confusionMatrix}\NormalTok{(knn.pred.train, train.norm.df[, }\DecValTok{14}\NormalTok{])}\OperatorTok{$}\NormalTok{table}
\end{Highlighting}
\end{Shaded}

\begin{verbatim}
##           Reference
## Prediction High Low
##       High   46   9
##       Low     0 248
\end{verbatim}

\begin{Shaded}
\begin{Highlighting}[]
\CommentTok{#Accuracy}
\KeywordTok{confusionMatrix}\NormalTok{(knn.pred.train, train.norm.df[, }\DecValTok{14}\NormalTok{])}\OperatorTok{$}\NormalTok{overall[}\DecValTok{1}\NormalTok{] }
\end{Highlighting}
\end{Shaded}

\begin{verbatim}
## Accuracy 
## 0.970297
\end{verbatim}

\subsubsection{Question 5}\label{question-5}

If the purpose is to predict CAT.MEDV for several billions of new
tracts, what would be the disadvantage of using k-NN prediction? List
the operations that the algorithm goes through in order to produce each
prediction.

The k-NN algorithm to make each prediction: (1) Calculate the distance
between a new record and every record in the training set; (2) Based on
the result of (1), determine the k-nearby records (k-nearest neighbors)
of the new record; (3) Classify the new record as the predominant class
among the k-nearby records.

Therefore, if we have several billions of new tracts, we need to repeat
steps (1) \textasciitilde{} (3) billions of times. This might take long
since step (1) requires much calculation time if we have a large
training set.


\end{document}
